\subsection{Benchmarks}
Visit \url{http://zlab.umassmed.edu/benchmark/} for a benchmark on various docking problems that are ordered by difficulty. There are test cases for rigid docking as well as flexible docking, again ordered by amount of typically changing parameters like torsion angles.

\subsection{PDB}
The \textit{Protein Data Bank Format} is a plain text data file, storing (amongst other information) the coordinates of every single atom in a protein.

A typical \textit{PDB}-file's row might look like this:

$$
\text{ATOM} 
\underbrace{2}_{\text{\begin{sideways}atom number\end{sideways}}}
\underbrace{\dots}_{\text{\begin{sideways}various identifiers\footnote{These identifiers are not of importance for this project.}\end{sideways}}}
\underbrace{57.827}_{\text{\begin{sideways}x-coordinate (\AA)\end{sideways}}}
\underbrace{\dots}_{\text{\begin{sideways}y- and z-coordinates\end{sideways}}}
\underbrace{1.00}_{\text{\begin{sideways}occupancy\footnote{In case an atom has been found in more than one locations all the different coordinates and according occupancies are listed—e.g. if it has been found in two different locations, both locations are represented as distinct rows with an occupancy factor of 0.5 each.}\end{sideways}}}
\underbrace{66.32}_{\text{\begin{sideways}temperature factor\footnote{The temperature factor is an indication of how rigidly the atom is held in place in the given structure. Lower values imply that it is a more ``stable'' position.}\end{sideways}}}
\underbrace{\text{N}}_{\text{\begin{sideways}element symbol\end{sideways}}}
$$
\todo{why aren't footnotes showing?}

For a list of PDB files for various proteins visit \url{http://www.rcsb.org/}.

\subsection{Modelling the Potential}
Given two metastable molecules $M_i=(\mathbf{x_i},\alpha_i,\beta_i,\gamma_i,\{\mathbf{a_{ij}}\}_{j=1}^{n_i})$, $i\in\{1,2\}$. In the following $\mathbf{x_i}\in\mathbb R^3$ will be called the molecule's (or the system's) point of reference, $\alpha_i,\beta_i,\gamma_i\in [0,2\pi)$ the system's rotation (where $\alpha_i$ corresponds to the rotation around the $x$-axis relative to its point of reference; analogously for $\beta_i, \gamma_i$) and $\mathbf{a_{ij}}\in\mathbb R^3$ the system's atoms with (absolute) coordinates $\mathbf{x_i}+\mathbf{a_{ij}}$, where $n_i$ is the number of atoms.

The Lennard-Jones potential of a system consisting of two points $\mathbf{a,b}\in\mathbb R^3$ with distance $r^2=||a-b||$ is defined as

\begin{equation}\label{LennardJones}
	V_{LJ}(r) = \varepsilon \left(\left(\frac{\sigma}{r}\right)^{12} - 2 \left(\frac{\sigma}{r}\right)^6\right),
\end{equation}

where $\varepsilon$ is the size of the potential well and $\sigma$ the distance at which the potential reaches its minimum. Both are usually given by experiments.

In order to investigate the energy landscape for a docking problem it is natural to consider the Lennard-Jones potential for a system consisting of two proteins $M_1,M_2$ given by

\begin{equation}\label{LennardJonesOf2Proteins}
	\tilde V_{LJ}(M_1,M_2) = \sum_{i=1}^{n_1+n_2}\sum_{j>i}^{n_1+n_2} V_{LJ}(r_{ij})
\end{equation}

if we set $r_{ij}$ to be the distance between the $i$th and $j$th atom\footnote{$i,j\in\{1,\dots,n_1+n_2\}$}.

If we want to explicitly calculate its gradient with respect to the free parameters $\mathbf{x_1},\alpha_1,\beta_1,\gamma_1$ it is instructive to write down the entire formula as a function of those variables.
First of all, to faciliate computation, we split up the summation into three parts $\tilde V_{LJ}(M_1,M_2) = \tilde V_{LJ}^1(M_1)+\tilde V_{LJ}^2(M_2)+\tilde V_{LJ}^3(M_1,M_2)$, where $\tilde V_{LJ}^1$ refers to all combinations of atoms of $M_1$, $\tilde V_{LJ}^2$ to those of $M_2$ and $\tilde V_{LJ}^3$ to the summation over all pairs $(a_{1i},a_{2j})_{i\in\{1,\dots,n_1\}, j\in\{1,\dots,n_2\}}$.
It is apparent that $\tilde V_{LJ}^1$ and $\tilde V_{LJ}^2$ do not change as a function of the arguments chosen above (however, they would in case we also introduced variable torsion angles and such).
To calculate the gradient it thus suffices to inspect $\tilde V_{LJ}^3$ as a function of the free parameters $x,y,z,\alpha,\beta,\gamma$ describing the position of $M_1$:

\begin{equation}
	\tilde V_{LJ}^3(\mathbf x_1,\alpha_1,\beta_1,\gamma_1) = \varepsilon 
	\sum_{i=1}^{n_1}\sum_{j=1}^{n_2}\left(\frac{\sigma}{r_{ij}}\right)^{12} 
	- 2 \left(\frac{\sigma}{r_{ij}}\right)^6
\end{equation}

where $r_{ij}^2 = \|\tilde a_{1i}-a_{2j}\|$ and $\tilde a_{1i}$ are the first system's $i$th particle's absolute coordinates given by

$$R_x(\alpha_1)R_y(\beta_1)R_z(\gamma_1)\mathbf a_{1i} + \mathbf x_1,$$ where $$R_x(\alpha_1) = \left(\begin{array}{ccc}
1 & 0 & 0 \\
0 & \text{cos}\,\alpha_1 & -\text{sin}\,\alpha_1 \\
0 & \text{sin}\,\alpha_1 &  \text{cos}\,\alpha_1 \end{array}\right),$$ $$R_y(\beta_1)=\left(\begin{array}{ccc}
\text{cos}\,\beta_1 & 0 & -\text{sin}\,\beta_1 \\
0 & 1 & 0\\
\text{sin}\,\beta_1 & 0 &  \text{cos}\,\beta_1\end{array}\right)$$ and $$R_z(\gamma_1) = \left(\begin{array}{ccc}
\text{cos}\,\gamma_1 & -\text{sin}\,\gamma_1 & 0 \\
\text{sin}\,\gamma_1 &  \text{cos}\,\gamma_1 & 0 \\
0 & 0 & 1 \end{array}\right).$$

Now one can explicitly compute the partial derivatives $\frac{\partial \tilde V_{LJ}}{\partial p} = \frac{\partial \tilde V_{LJ}^3}{\partial p}$, for $p$ being one of the parameters.

As of now we can compute the potential $\tilde V_{LJ}(\mathbf X)$ for a given point $\mathbf X = (\mathbf x_1,\alpha_1,\beta_1,\gamma_1)$ in the state space and its gradient $\nabla \tilde V_{LJ}(\mathbf X)$ and that is all we need for algorithms like steepest descent (which would still require rescaling of the parameters as to avoid extensive zig-zagging) or diffusion maps.

Furthermore, note that the computation of the potential and the gradient can be combined into one double loop, making\footnote{although the computation of the gradient is likely to result in a big constant factor of operations} it $O\left((n_1+n_2)^2\right)$.



