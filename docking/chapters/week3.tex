\subsection{Ordering Dihedral Angles by Variance}
To be able to approximate the real docking process as closely as possible while only letting some chosen dihedral angles be variable, one first needs to order them accordingly. One such order might be given by regarding each such angle $\Theta_i$ as a discrete random variable and ordering them by their variances.

More precisely, a set of a single molecule's uniformly distributed\footnote{This requirement is necessary because we do not want to count ``very similar'' conformations twice.} conformations $\{M_k\}_{k=0}^n$ induces a discrete probability space $(\{M_k\}_{k=0}^n,\mathscr{A},P)$ where $P(M_k):=\frac{V_{LJ}(M_k)}{Z}$, where $Z:=\sum_{k=1}^n P(M_k)$ is a normalising factor and for all $A$ in $\mathscr{A}: P(A) = \sum_{M\in A}P(M)$. 

Another approach might be to compute sample trajectories and use them (regardless of their distribution) instead of uniformly distributed samples.

For a fixed $\Theta_i$ we obtain the expected value $E[\Theta_i] = \sum_{k=1}^n P(M_k)\Theta_{ik}$, were $\Theta_{ik}$ is the $k$-th conformation's $i$-th dihedral angle. However, this approach does not consider the fact that these values are circularly distributed; thus it may be better to assume that the angles' values are normally distributed and to estimate the parameters of the von Mises distribution.

The variance $\text{Var}[\Theta_i]$ is given by $E[(\Theta_i - E[\Theta_i])^2]$.

Other ideas included:
\begin{itemize}
	\item Looking if such a list already existed for some molecules, but I couldn't find any.
	\item Taking different conformations from \url{rcsb.com} and using PCA on the phase space consisting of the dihedral angles, but given the number of dimensions compared to the amount of samples available\footnote{For albumin there are 104 entries as of 14.04.2015.} this is not a promising approach either.
	\item Take Ramachandran plots, interpret them as conditional probabilities and combine all of them. The data for this approach should be available, but combining all those single results may be hard.
\end{itemize}

\subsection{Rotation About an Arbitrary Axis in 3D}
To simulate flexible docking it is essential to be able to determine the single atoms' Cartesian coordinates, given one point of reference, the distance between two adjacent atoms, the angles of three consecutive atoms and the dihedral angles.

Given four atoms $\{a_i\}_{i=0}^3$, s.t. $a_j$ and $a_{j+1}$ are covalently bonded their dihedral angle is the smallest angle between the two planes $\pi_0$ and $\pi_1$ which are uniquely determined by $\{a_i\}_{i=0}^2$ and $\{a_i\}_{i=1}^3$, respectively.

Given $\{a_i\}_{i=0}^2$ one sets $a_3$ in $π_0$, s.t. it has the prescribed distance as well as the angle between $a_1,a_2$ and $a_3$.
Now we want to rotate $a_3$ about the line $a_2 + \overrightarrow{a_1-a_2}$.

Clearly this is not a linear mapping unless $a_2=0$, but by introducing a forth dimension we can turn three-dimensional translations into linear mappings.

See \url{http://inside.mines.edu/fs_home/gmurray/ArbitraryAxisRotation/} for a derivation of the according formula.





