\subsection{The Original Goal}
Originally I wanted to approach the problem of protein folding by investigating the asymptotic ($t\rightarrow\infty$) behaviour of the probability density function of a given protein. As the changes over time of a given protein\footnote{understood as a single point in high dimensional space} can be modelled by a Langevin equation $\dot x = -\nabla U(x)+\beta\dot w$, the corresponding probability density function $p(t,x)$ is described by the forward Fokker-Planck equation $\frac{\partial p}{\partial t} = \tilde\beta \Delta p + \text{div}(p\nabla U)$ whose asymptotic behaviour has been investigated in \cite{Nadler2008}.

However, there are (at least) two reasons why this does not work:
\begin{itemize}
	\item In order to get a meaningful low dimensional representation of the system, a considerable margin between two adjacent eigenvalues $\lambda_{k+1} \gg \lambda_k$ of the Fokker-Planck operator is needed and it can be shown that this roughly corresponds to having a potential function $U$ with $k$ local minima\footnote{c.f. \cite[pp.9]{Nadler2008}, p.9}. Taking these properties into consideration it seems futile to employ this method of dimensionality reduction as the potential function in the area of protein folding has thousands of local minima.
	\item Even if solutions could be found explicitly in a reasonable amount of time the problem of finding and exact potential function $U$ still remains unsolved\footnote{c.f. \cite{Neumaier97}, p.14ff}. As we are working with approximations it does not seem to make any sense trying to study its asymptotic long-term behaviour.
\end{itemize}

As a result of above considerations my topic of this seminar was restricted to docking. In the special case of rigid docking both molecules are assumed to be in a metastable state which drastically cuts the the complexity of the state space.  